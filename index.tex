% Options for packages loaded elsewhere
\PassOptionsToPackage{unicode}{hyperref}
\PassOptionsToPackage{hyphens}{url}
\PassOptionsToPackage{dvipsnames,svgnames,x11names}{xcolor}
%
\documentclass[
  letterpaper,
  DIV=11,
  numbers=noendperiod]{scrreprt}

\usepackage{amsmath,amssymb}
\usepackage{lmodern}
\usepackage{iftex}
\ifPDFTeX
  \usepackage[T1]{fontenc}
  \usepackage[utf8]{inputenc}
  \usepackage{textcomp} % provide euro and other symbols
\else % if luatex or xetex
  \usepackage{unicode-math}
  \defaultfontfeatures{Scale=MatchLowercase}
  \defaultfontfeatures[\rmfamily]{Ligatures=TeX,Scale=1}
\fi
% Use upquote if available, for straight quotes in verbatim environments
\IfFileExists{upquote.sty}{\usepackage{upquote}}{}
\IfFileExists{microtype.sty}{% use microtype if available
  \usepackage[]{microtype}
  \UseMicrotypeSet[protrusion]{basicmath} % disable protrusion for tt fonts
}{}
\makeatletter
\@ifundefined{KOMAClassName}{% if non-KOMA class
  \IfFileExists{parskip.sty}{%
    \usepackage{parskip}
  }{% else
    \setlength{\parindent}{0pt}
    \setlength{\parskip}{6pt plus 2pt minus 1pt}}
}{% if KOMA class
  \KOMAoptions{parskip=half}}
\makeatother
\usepackage{xcolor}
\setlength{\emergencystretch}{3em} % prevent overfull lines
\setcounter{secnumdepth}{5}
% Make \paragraph and \subparagraph free-standing
\ifx\paragraph\undefined\else
  \let\oldparagraph\paragraph
  \renewcommand{\paragraph}[1]{\oldparagraph{#1}\mbox{}}
\fi
\ifx\subparagraph\undefined\else
  \let\oldsubparagraph\subparagraph
  \renewcommand{\subparagraph}[1]{\oldsubparagraph{#1}\mbox{}}
\fi


\providecommand{\tightlist}{%
  \setlength{\itemsep}{0pt}\setlength{\parskip}{0pt}}\usepackage{longtable,booktabs,array}
\usepackage{calc} % for calculating minipage widths
% Correct order of tables after \paragraph or \subparagraph
\usepackage{etoolbox}
\makeatletter
\patchcmd\longtable{\par}{\if@noskipsec\mbox{}\fi\par}{}{}
\makeatother
% Allow footnotes in longtable head/foot
\IfFileExists{footnotehyper.sty}{\usepackage{footnotehyper}}{\usepackage{footnote}}
\makesavenoteenv{longtable}
\usepackage{graphicx}
\makeatletter
\def\maxwidth{\ifdim\Gin@nat@width>\linewidth\linewidth\else\Gin@nat@width\fi}
\def\maxheight{\ifdim\Gin@nat@height>\textheight\textheight\else\Gin@nat@height\fi}
\makeatother
% Scale images if necessary, so that they will not overflow the page
% margins by default, and it is still possible to overwrite the defaults
% using explicit options in \includegraphics[width, height, ...]{}
\setkeys{Gin}{width=\maxwidth,height=\maxheight,keepaspectratio}
% Set default figure placement to htbp
\makeatletter
\def\fps@figure{htbp}
\makeatother
\newlength{\cslhangindent}
\setlength{\cslhangindent}{1.5em}
\newlength{\csllabelwidth}
\setlength{\csllabelwidth}{3em}
\newlength{\cslentryspacingunit} % times entry-spacing
\setlength{\cslentryspacingunit}{\parskip}
\newenvironment{CSLReferences}[2] % #1 hanging-ident, #2 entry spacing
 {% don't indent paragraphs
  \setlength{\parindent}{0pt}
  % turn on hanging indent if param 1 is 1
  \ifodd #1
  \let\oldpar\par
  \def\par{\hangindent=\cslhangindent\oldpar}
  \fi
  % set entry spacing
  \setlength{\parskip}{#2\cslentryspacingunit}
 }%
 {}
\usepackage{calc}
\newcommand{\CSLBlock}[1]{#1\hfill\break}
\newcommand{\CSLLeftMargin}[1]{\parbox[t]{\csllabelwidth}{#1}}
\newcommand{\CSLRightInline}[1]{\parbox[t]{\linewidth - \csllabelwidth}{#1}\break}
\newcommand{\CSLIndent}[1]{\hspace{\cslhangindent}#1}

\KOMAoption{captions}{tableheading}
\makeatletter
\makeatother
\makeatletter
\@ifpackageloaded{bookmark}{}{\usepackage{bookmark}}
\makeatother
\makeatletter
\@ifpackageloaded{caption}{}{\usepackage{caption}}
\AtBeginDocument{%
\ifdefined\contentsname
  \renewcommand*\contentsname{Table of contents}
\else
  \newcommand\contentsname{Table of contents}
\fi
\ifdefined\listfigurename
  \renewcommand*\listfigurename{List of Figures}
\else
  \newcommand\listfigurename{List of Figures}
\fi
\ifdefined\listtablename
  \renewcommand*\listtablename{List of Tables}
\else
  \newcommand\listtablename{List of Tables}
\fi
\ifdefined\figurename
  \renewcommand*\figurename{Figure}
\else
  \newcommand\figurename{Figure}
\fi
\ifdefined\tablename
  \renewcommand*\tablename{Table}
\else
  \newcommand\tablename{Table}
\fi
}
\@ifpackageloaded{float}{}{\usepackage{float}}
\floatstyle{ruled}
\@ifundefined{c@chapter}{\newfloat{codelisting}{h}{lop}}{\newfloat{codelisting}{h}{lop}[chapter]}
\floatname{codelisting}{Listing}
\newcommand*\listoflistings{\listof{codelisting}{List of Listings}}
\makeatother
\makeatletter
\@ifpackageloaded{caption}{}{\usepackage{caption}}
\@ifpackageloaded{subcaption}{}{\usepackage{subcaption}}
\makeatother
\makeatletter
\@ifpackageloaded{tcolorbox}{}{\usepackage[many]{tcolorbox}}
\makeatother
\makeatletter
\@ifundefined{shadecolor}{\definecolor{shadecolor}{rgb}{.97, .97, .97}}
\makeatother
\makeatletter
\makeatother
\ifLuaTeX
  \usepackage{selnolig}  % disable illegal ligatures
\fi
\IfFileExists{bookmark.sty}{\usepackage{bookmark}}{\usepackage{hyperref}}
\IfFileExists{xurl.sty}{\usepackage{xurl}}{} % add URL line breaks if available
\urlstyle{same} % disable monospaced font for URLs
\hypersetup{
  pdftitle={Applied Programming for Geosciences},
  pdfauthor={Maximilian Nölscher},
  colorlinks=true,
  linkcolor={blue},
  filecolor={Maroon},
  citecolor={Blue},
  urlcolor={Blue},
  pdfcreator={LaTeX via pandoc}}

\title{Applied Programming for Geosciences}
\usepackage{etoolbox}
\makeatletter
\providecommand{\subtitle}[1]{% add subtitle to \maketitle
  \apptocmd{\@title}{\par {\large #1 \par}}{}{}
}
\makeatother
\subtitle{From Data Cleaning to Publishing}
\author{Maximilian Nölscher}
\date{3/2/23}

\begin{document}
\maketitle
\ifdefined\Shaded\renewenvironment{Shaded}{\begin{tcolorbox}[interior hidden, frame hidden, borderline west={3pt}{0pt}{shadecolor}, boxrule=0pt, sharp corners, breakable, enhanced]}{\end{tcolorbox}}\fi

\renewcommand*\contentsname{Table of contents}
{
\hypersetup{linkcolor=}
\setcounter{tocdepth}{2}
\tableofcontents
}
\bookmarksetup{startatroot}

\hypertarget{content}{%
\chapter*{Content}\label{content}}
\addcontentsline{toc}{chapter}{Content}

\markboth{Content}{Content}

Welcome to \emph{Applied R Programming for Geoscience --- From Data
Cleaning to Publishing}! This online book contains the material for the
corresponding R teaching course. The content of this course focusses on
showing the capabilities of programming languages (in this cas R) for
every task throughout most common in research project workflows:
starting with

\begin{itemize}
\tightlist
\item
  data import, cleaning and wrangling,
\item
  analysis,
\item
  high quality visualizations to
\item
  writing manuscripts for peer-reviewed journal publishing.
\end{itemize}

all done in one single free open-source environment. Due to the
contrained time, this course is not meant to be a comprehensive
programming with R introductory course and therefore, will not go into
details of each of the mentioned steps. It will rather show that all
this can be done in R. I hope that seeing the advantages of it will
encourage all participants to further keep learning programming and
applying all the useful tools that come with it. As compensation, this
book provides hints to online resources that help to get a foot into the
door of improving R programming skills.

\hypertarget{acknowledgements}{%
\section*{Acknowledgements}\label{acknowledgements}}
\addcontentsline{toc}{section}{Acknowledgements}

\markright{Acknowledgements}

\bookmarksetup{startatroot}

\hypertarget{introduction-li-dxoycpzg}{%
\chapter{Introduction \{\{\textless=``\,'' li=``\,'' dxoycpzg=``\,''
\textgreater\}=``\,``\}}\label{introduction-li-dxoycpzg}}

\hypertarget{motivation}{%
\section{Motivation}\label{motivation}}

During my last few years as early career scientist in hydrogeology, I
noticed something, that has been studied and written about a lot
already: Reproducibility! Reproducibility in science! And there are even
concepts and tools for adressing this issue and to ensure a high level
of reproducibility in science since the early days of computers and
programming.

\hypertarget{readership-who-is-this-for}{%
\section{Readership (who is this
for?)}\label{readership-who-is-this-for}}

This course or book is meant for anybody

\begin{itemize}
\tightlist
\item
  interested in learning essential \textbf{tools for data science}
\item
  interested in state-of-the-art methods for \textbf{reproducibility in
  science}
\item
  \textbf{with or without} prior knowledge of programming in general or
  R (although it wouldn't hurt of course)
\end{itemize}

\hypertarget{about-me}{%
\section{About Me}\label{about-me}}

I am working as a researcher in the field of machine learning and
hydrogeology at the Federal Institute for Geoscience and Natural
Resources (BGR). My projects include modelling groundwater quality
parameters for producing maps as well as modelling groundwater level
time series for making predictions. For this purpose I startet learning
R mainly and a little bit Python a few years ago. Besides this, I enjoy
using R for making data visualizations and teaching R and its
capabilities to others.

\bookmarksetup{startatroot}

\hypertarget{r-basics}{%
\chapter{R Basics}\label{r-basics}}

\hypertarget{overview-and-terms}{%
\section{Overview and Terms}\label{overview-and-terms}}

sada

\begin{itemize}
\tightlist
\item
  R
\item
  RStudio
\item
  git
\item
  github
\item
  quarto
\item
  Packages
\item
  Function
\item
  Vignette
\end{itemize}

\hypertarget{data-types}{%
\section{Data types}\label{data-types}}

text.

\hypertarget{further-resources}{%
\section{Further Resources}\label{further-resources}}

\hypertarget{programming-with-r}{%
\subsection{Programming with R}\label{programming-with-r}}

\begin{itemize}
\tightlist
\item
  Learning fundamental programming in R for data science with the
  \texttt{tidyverse}
  meta-package:\href{https://r4ds.had.co.nz/index.html}{R for Data
  Science \textbar{} by Hadley Wickham and Garret Grolemund}
\item
  Introduction into
  R:\href{https://livebook.manning.com/book/r-in-action-third-edition/r-in-action/4}{R
  in Action \textbar{} by Robert I. Kabacoff} (only partially available
  for free)
\item
  In-depth best practices for efficient R
  programming:\href{https://csgillespie.github.io/efficientR/}{Efficient
  R programming \textbar{} by Colin Gillespie and Robin Lovelace}
\item
  Wrangling, analyzing, and vizualising geo
  data:\href{https://geocompr.robinlovelace.net/index.html}{Geocomputation
  with R \textbar{} by Robin Lovelace, Jakub Nowosad and Jannes
  Muenchow}
\item
  Handling spatial data in R:\href{https://r-spatial.org/book/}{Spatial
  Data Science \textbar{} by Edzer Pebesma and Roger Bivand}
\item
  All you need to know about R with git and
  github:\href{https://happygitwithr.com/index.html}{Happy Git and
  GitHub for the useR \textbar{} by Jenny Bryan}
\item
  Basics of machine learning in R using the \texttt{tidymodels}
  meta-package:\href{https://www.tmwr.org/}{Tidy Modeling with R
  \textbar{} by Max Kuhn and Julia Silge}
\item
  Introduction and in-depth knowledge for writing R
  packages:\href{https://r-pkgs.org/}{R Packages \textbar{} by Hadley
  Wickham and Jenny Bryan}
\end{itemize}

\hypertarget{learning-r-through-challenges-and-small-projects}{%
\subsection{Learning R --- through Challenges and small
Projects}\label{learning-r-through-challenges-and-small-projects}}

For speeding up the learning process it is more important to practise
rather than reading books. There are many great places in the internet
where you can find smaller and bigger data science challenges and
projects. I really recommend to visit those places and just learn by
directly solving tasks. Most important ones are:

\begin{itemize}
\item
  Kaggle
\item
  TidyTuesday
\item
  AdventOfCode
\item
  \href{https://30daymapchallenge.com/}{30DayMapChallenge}
\end{itemize}

\hypertarget{community}{%
\subsection{Community}\label{community}}

The R community is very friendly and supportive community. Most
important packages have great documentations for learning new R packages
and frameworks. The R community has an extraordinary sense of sharing
knowledge and experiences. Important platforms for getting engaged with
the R community but also for getting help with questions, issues and
bugs are

\begin{itemize}
\item
  \href{https://stackoverflow.com/questions/tagged/r}{Stackoverflow} for
  getting help with questions
\item
  \href{https://community.rstudio.com/}{Posit community forum (formerly
  Rstudio)} for getting help with questions
\item
  \href{https://twitter.com/search?q=\%23rstats\&src=typed_query}{Twitter
  with \#rstats} for keeping track of latest developments, tips, tricks
  and getting in touch with the active R community members
\item
  \href{https://mastodon.social/}{Mastodon} with \#rstats, same same
\item
  \href{https://github.com/}{Github} for getting help with bugs, getting
  to know the developers behind packages or simply hosting and
  versioning your own code
\end{itemize}

\bookmarksetup{startatroot}

\hypertarget{the-data}{%
\chapter{The Data}\label{the-data}}

\hypertarget{import}{%
\section{Import}\label{import}}

\hypertarget{cleaning}{%
\section{Cleaning}\label{cleaning}}

\hypertarget{imputation}{%
\section{Imputation}\label{imputation}}

\hypertarget{wrangling}{%
\section{Wrangling}\label{wrangling}}

\bookmarksetup{startatroot}

\hypertarget{exploratory-data-analysis}{%
\chapter{Exploratory Data Analysis}\label{exploratory-data-analysis}}

\hypertarget{missing-data}{%
\section{Missing Data}\label{missing-data}}

\hypertarget{distributions-and-correlations}{%
\section{Distributions and
Correlations}\label{distributions-and-correlations}}

\bookmarksetup{startatroot}

\hypertarget{visualizations}{%
\chapter{Visualizations}\label{visualizations}}

This is a very helpful tool for finding out the best suited chart type
for a data visualization of a given data structure:

\bookmarksetup{startatroot}

\hypertarget{the-data-1}{%
\chapter{The Data}\label{the-data-1}}

\hypertarget{import-1}{%
\section{Import}\label{import-1}}

\hypertarget{cleaning-1}{%
\section{Cleaning}\label{cleaning-1}}

\hypertarget{imputation-1}{%
\section{Imputation}\label{imputation-1}}

\hypertarget{wrangling-1}{%
\section{Wrangling}\label{wrangling-1}}

\bookmarksetup{startatroot}

\hypertarget{references}{%
\chapter*{References}\label{references}}
\addcontentsline{toc}{chapter}{References}

\markboth{References}{References}

\hypertarget{refs}{}
\begin{CSLReferences}{0}{0}
\end{CSLReferences}



\end{document}
